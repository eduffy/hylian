

The software maintenance process, including comprehension, modification,
and refactoring of complex multi-par\-a\-digm systems, requires 
extensive and detailed information about the system under study.
However, software artifacts that provide this information are
frequently unavailable and for large, open-source applications,
they are virtually nonexistent.
Thus, much of the research in software maintenance has focused 
on the development of inquiry and analysis tools to automate the
process of generating information to improve comprehension, and to 
facilitate analysis, modification and testing of the application under 
study.

However, the {\CPP} language has proven to be particularly problematic 
for maintenance engineers interested in developing tools to facilitate
analysis and modification of {\CPP} applications.
The difficulty in developing tools for {\CPP} is mostly 
due to the scope and complexity of the language; for example,
the grammatical representation of {\CPP} has been shown to be  
larger and more complex than other, commonly used
languages \cite{Power-Malloy:04}.
A particularly perplexing problem for {\CPP} maintenance engineers
 entails the correct recognition of the language constructs
as specified in the ISO standard, for example class template partial
specializations and argument-dependent lookup \cite[\S A.8]{C++standard98}.
Moreover, statement-level information,
which is required for pointer analysis and program 
slicing \cite{binkley2006,gallagher2003,Harrold1996MAY,Weiser84},
relies on the correct recognition of expressions such
as {\em expression}, {\em postfix-expression},
and {\em unary-expression} \cite[\S A.4]{C++standard98}.

Nevertheless, the {\CPP} language is frequently used and recently
has been shown to outperform other, commonly used, languages
by a large margin \cite{preez2011}.
Therefore, to support software maintenance and other software
engineering efforts for the {\CPP} language, it's important to
develop analysis tools for the language.
Several research projects have attempted to address the problem
of providing analysis of {\CPP} language constructs, including
Elkhound, {\sf SourceNavigator}\texttrademark, {\sf Columbus},
{\gfourre}, and libclang 
\cite{elsa2007,sourcenav,columbus2005,kraft05-wcre,kraft06-ist,www-libclang}.
However, due to the difficulties in parsing the {\CPP} language,
these projects are unable to provide statement level analysis.
Therefore, they are unable to regenerate the source code, which
precludes dynamic analysis, and they cannot provide details
about statements and expressions, which precludes analysis such
as pointer alias and slicing.

In this paper, we describe our extension of a system, Hylian,
which utilizes the {\em gcc} parser to provide statement level
analysis of {\CPP} programs. The initial phases of Hylian were 
described in references 
\cite{duffy05-sera,duffy-wcre07,Duffy-Malloy:2007,Duffy-etal:2008,IJSEKE}.
In this paper, we describe our construction of an abstract semantic 
graph (ASG), that is {\em language-complete}, which is a semantic graph
that includes all aspects defined in the language standard.
Our focus is an ASG that is language-complete
for the 2003 revised ISO {\CPP} language standard; our current
implementation does not include the {\CPP}11 standard, recently
ratified \cite{isocpp-11}.
A semantic graph that completely defines {\CPP} must include:
evaluation and lookup
of constants, full type resolution for names, determination of
type equivalency and type promotion, full and partial template
instantiation, operator overload resolution, function and method
resolution including argument dependent name lookup, implicit
method invocation commonly used in class construction and destruction.
The current implementation of Hylian provides information about
all of these constructs.

In the next section, we review projects that relate to
our work, and in Section \ref{sec:overview} we provide an overview 
of the three phases of Hylian. In Section \ref{sec:method} we 
provide details about construction of the Hylian ASG, and
in Section \ref{sec:results} we list statistics that 
summarize efficiency considerations for ASG construction, including 
some size results for an ASG for a popular video game. 
In Section \ref{sec:conclude} we draw conclusions.


