
In this section, we review the research that facilitates static or 
dynamic analysis of {\CPP} language constructs, from high level
constructs such as classes and class hierarchies, to low level
constructs, such as statements or expressions.
Statement and expression analysis requires a parser front-end that
is language complete for the ISO standard, and can syntactically
and semantically represent all constructs in the language. 
%In the case of the {\CPP} language, these constructs include
%full type resolution for names, determination of
%type equivalency and type promotion, full and partial template
%instantiation, operator overload resolution, function and method
%resolution including argument dependent name lookup, and implicit
%method invocation commonly used in class construction and destruction.

We only review the work with a focus on analysis of {\CPP}
programs, and that make this analysis information available to
a developer, typically in graph form.
For example, Elkhound is a parser generator that uses the Generalized 
LR (GLR) parsing algorithm \cite{elsa2007,tomita85,vandenbrand98}.
The Elkhound parser, Elsa, attempts to accommodate several dialects 
of {\CPP} including {\em gcc} and Visual {\CPP} \cite{elsa2007}.
Elsa can accommodate most of the {\CPP} grammar,
but does not fully accommodate either dialect.
For example, only 89\% of the test cases in the {\em gcc} test suite, 
were parsed by Elsa. However, for {\Fluxbox}, a lightweight
window manager for X, only 32 of the 74 translation units were 
parsed by Elsa \cite{www-fluxbox}.  Also, the Elsa project has 
not been recently updated \cite{elsa2007}.
By contrast, Hylian can parse all test cases in the {\em gcc} test
suite and we have been able to provide analysis of the {\Fluxbox}
application \cite{kraft-tse}. 
To make the analysis information available, Hylian
builds an abstract semantic graph (ASG) with subgraphs for all
{\CPP} language constructs, including instantiation of templates.

{\sf SourceNavigator}\texttrademark is an analysis and graphical 
framework from Red Hat \cite{sourcenav}. SourceNavigator 
accepts C, {\CPP}, Java, Tcl, FORTRAN, and COBOL, and the 
accompanying fuzzy parser extracts enough high level information to 
provide class hierarchies, imprecise call graphs, and include graphs.
However, since {\sf SourceNavigator} employs a fuzzy parse, it
does not provide statement level information.

{\sf Columbus} is an integrated reverse engineering framework 
supporting fact extraction, linking, and analysis for C and {\CPP} 
programs \cite{columbus2005}. 
{\sf Columbus} provides output in a variety of formats, including
{CPPML} (OCaml-like pattern-matching extensions for {\CPP}),
{GXL} (a schema for storing graphs in XML), {RSF} (a reverse-engineering
tool that supports source code transformations for {\CPP} applications),
and {XMI} (an Object Manager Group metadata standard for XML).
Nevertheless, {\sf Columbus} is unable to fully accept templates,
as noted in reference \cite{tuanalyzer04}.


The {\gfourre} system, like Hylian, uses {\em gcc} as its parser
front-end. {\gfourre} is designed as a tool chain, and consists 
of applications and libraries that can be used either individually 
or as a single unit \cite{kraft05-wcre,kraft06-ist}.
The implementation of {\gfourre} uses  a GXL-based pipe-filter 
architecture, including a set of loosely coupled, reusable 
modules including an {\em ASG module}.
Advantages of {\gfourre} include an API to provide easy access to
information about declarations, including classes, functions,
and variables, as well as information about scopes, types, and
control statements. 
However, {\gfourre} is not code-complete, so that 
the original source code cannot be regenerated from
the {\em ASG module}. 
Thus, the {\gfourre} system is limited to static analysis of
type information and cannot provide information about statements
or expressions.

An alternative, open source library for parsing {\CPP} language
constructs includes clang \cite{www-clang}, and its C interface, 
libclang \cite{www-libclang};
Thus, clang is the parser front-end used by libclang; 
the goal of libclang is to facilitate IDE development,
most notably, Apple's XCode.  The libclang API can
track variable declarations, provide syntax highlighting, code 
completion, symbol renaming, and cross-referencing, including 
references to global variables across compilation units. 
However, even though information about statements is complete in
clang, libclang does not present all of this information
to clients of the API.
However, like libclang and {\em gcc}, this information is not 
readily available, so that an interested developer
must investigate clang internals, whose documentation exists in 
the form of source code comments and developer mailing lists.
Thus, the complete information about statements known to clang
is not accessible to a libclang user, either in graph form, or as
an easily accessible API.

